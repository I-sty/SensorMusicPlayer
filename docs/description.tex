\documentclass[12pt, a4paper, portrait]{article}
%Gummi|065|=)
\usepackage[affil-it]{authblk}
\usepackage{hyperref}
\hypersetup{
  colorlinks=true,
  linkcolor=black,
  filecolor=magenta,
  urlcolor=blue
}
\urlstyle{same}

\usepackage{listings}
\usepackage{color}
\definecolor{codegreen}{rgb}{0,0.6,0}
\definecolor{codegray}{rgb}{0.5,0.5,0.5}
\definecolor{codepurple}{rgb}{0.58,0,0.82}
\definecolor{backcolour}{rgb}{0.95,0.95,0.92}
\lstdefinestyle{mystyle}{
    backgroundcolor=\color{backcolour},   
    commentstyle=\color{codegreen},
    keywordstyle=\color{magenta},
    numberstyle=\tiny\color{codegray},
    stringstyle=\color{codepurple},
    basicstyle=\footnotesize,
    breakatwhitespace=false,         
    breaklines=true,                 
    captionpos=b,                    
    keepspaces=true,                 
    numbers=left,                    
    numbersep=5pt,                  
    showspaces=false,                
    showstringspaces=false,
    showtabs=false,                  
    tabsize=2
}
 
\lstset{style=mystyle}

\usepackage{indentfirst}
\usepackage[utf8]{inputenc}

\usepackage{geometry}
\geometry{
 a4paper,
 left=20mm,
 top=20mm,
 right=25mm,
 bottom=20mm
}

% Personal data
\title{\textbf{Sensor Music Player}}
\author{István Szőllősi}
\affil{Faculty of Sciences and Letters, ``Petru Maior'' University of Târgu Mureș}
\date{August 25, 2018}

\renewcommand{\baselinestretch}{1.5} 
\begin{document}

\maketitle
\newpage

\tableofcontents
\newpage

\section{About}
The project is committed to the GitHub, you can find \href{https://github.com/I-sty/SensorMusicPlayer/}{here}.
\par The main structure of the repository is
\textit{a valid Android project} with several additionals folders, like the:
\begin{itemize}
\item{\textbf{backend}} folder where the \textit{Python} and \textit{JavaScript} codes are stored
\item{\textbf{docs}} folder where the documents about the project are stored
\end{itemize}
\section{Node.js}
In Node.js is very simple to create a small web server for REST calls.
\subsection{Installation}
\subsection{Configuration}
Used tutorial: \href{https://www.codementor.io/olatundegaruba/nodejs-restful-apis-in-10-minutes-q0sgsfhbd}{Build Node.js RESTful APIs in 10 Minutes}
\subsubsection{Mongoose}
\subsubsection{Express}
\subsubsection{Nodemon}

\section{MongoDB}
MongoDB to store signal data from the \textit{Y axis} of the accelerometer from the Android devices.
\subsection{Drop collection}
Code:
\begin{lstlisting}[language=bash, caption=MongoDB shell commands to drop a collection]
show dbs
use <db>
show collections
db.<collection>.drop()
\end{lstlisting}

\section{Python PyPlot}
\href{https://matplotlib.org/users/installing.html#linux-using-your-package-manager}{Install library from here}

\section{Postman}
\subsection{Installation}
Installed according to this article: \href{https://r00t4bl3.com/post/how-to-install-postman-native-app-in-linux-mint-18-3-sylvia}{How to install Postman native app in Linux Mint 18.3}

Used to test the main functionalities of the Node.js server.
\subsection{Usage}
To get all buffer paste this code in Postman:
\begin{lstlisting}[language=bash, caption=Get all buffers]
curl -X GET http://localhost:3000/buffers
\end{lstlisting}

The response is or an empty list, if no items in the database or a list like this:
\begin{lstlisting}[language=XML, caption=A sub section of the signal to process]
[
    {
        "value": [
            5.733050346374512,
            1.704751968383789,
            -2.7134790420532227,
            -1.343064308166504,
            2.6042985916137695,
            3.92281436920166,
            2.15725040435791,
            -0.9106369018554688,
            -2.4146032333374023,
            -2.943338394165039,
            -1.5269522666931152,
            -0.8230304718017578,
        ],
        "_id": "5b82607f5601ec575d3bf0e4",
        "__v": 0
    }
]
\end{lstlisting}

\end{document}
